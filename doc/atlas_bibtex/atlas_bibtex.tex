%-------------------------------------------------------------------------------
% Examples on how to fill BibTeX entries in the ATLAS Bibliographic Style
% Responsible: Ian Brock (Ian.Brock@cern.ch)
%-------------------------------------------------------------------------------
\documentclass[UKenglish]{latex/atlasdoc}

% Use biblatex and biber for the bibliography
%\usepackage[minimal]{latex/atlaspackage}
\usepackage[minimal,backend=biber]{latex/atlaspackage}
\usepackage[articletitle=true,showdoi=false]{latex/atlasbiblatex}

\usepackage{authblk}
\usepackage{latex/atlasphysics}

% Files with references in BibTeX format
\addbibresource{atlas_bibtex.bib}
\addbibresource{atlas_biblatex.bib}
\addbibresource{../atlas_latex.bib}

\graphicspath{{../../logos/}}

\newcommand*{\BibTeX}{Bib\TeX}
\newcommand{\File}[1]{\texttt{#1}\xspace}
\newcommand{\Macro}[1]{\texttt{\textbackslash #1}\xspace}
\newcommand{\Option}[1]{\textsf{#1}\xspace}
\newcommand{\Package}[1]{\texttt{#1}\xspace}

%-------------------------------------------------------------------------------
% Generic document information
%-------------------------------------------------------------------------------

% Set author and title for the PDF file
\hypersetup{pdftitle={ATLAS BibTeX guide},pdfauthor={Ian Brock}}

\AtlasTitle{Guide to references and \BibTeX\ in ATLAS documents}

\author{Ian C. Brock}
\affil{University of Bonn}

\AtlasAbstract{%
  This document illustrates how to use \BibTeX\ for the bibliography of your ATLAS paper.
  It discusses what to pay attention to when creating references and how to get round common problems.
  Use of both \Package{biblatex} and the more traditional \BibTeX\ is covered.
  Two \BibTeX\ (\texttt{.bst}) style files have been created that can be used with any of the ATLAS supported journals,
  depending on whether they require the title of the references to be included or not.

  This document was generated using version \ATPackageVersion\ of the ATLAS \LaTeX\ package.
}


%-------------------------------------------------------------------------------
\begin{document} 

%\maketitle

%-------------------------------------------------------------------------------
\section{Introduction}
%-------------------------------------------------------------------------------

The ATLAS Collaboration has specific guidelines as to what constitutes a good bibliographic style. 
For example, a reference to a paper by an LHC Collaboration must not include the first author,
whereas, if the paper is by any other collaboration it should. 
Where available, links to the \texttt{arXiv} entries of the papers should be included. 
To help authors with their paper preparations,
a standard ATLAS bibliographic style has been developed which incorporates all of these requirements, 
and, at the same time, is largely compatible with those of the journals the papers are being submitted to. 

It is strongly recommended to use \BibTeX{} for the references. 
Although it often appears harder to use at the beginning, it means that the number of
typos should be reduced significantly and the format of the references
will be correct, without you having to worry about formatting it.
In addition the order of the references is automatically correct.

A new implementation of \BibTeX\ is provided by the \texttt{biblatex}~\cite{biblatex} package.
All ATLAS documents use this package now by default.
One major advantage of the package is that it defines quite a few more entry types
that are much more suitable for online documents and things like CONF and PUB notes.
It is also possible to use UTF-8 encoding in the entries, which means that letters such as
ä, é, ß can be included directly in the text.
Adjustment of the style is also much simpler.
It is possible to take a base style and then just apply changes to it rather than
having to learn the details of how \texttt{bst} files are constructed.

The typical compilation cycle when using \Package{biblatex} with the \BibTeX\ backend 
or the traditional version of \BibTeX{} looks like the following:
%
\begin{verbatim}
  pdflatex mydocument
  bibtex mydocument
  pdflatex mydocument
  pdflatex mydocument
\end{verbatim}
If you use \Package{biblatex} with the backend \Package{biber} rather than \Package{bibtex}, replace the command
\texttt{bibtex} with \texttt{biber}.

\Package{biblatex} and \BibTeX{} create a file with the extension \texttt{.bbl}, which
contains the actual references used, and \LaTeX{} then takes care
to include them in your paper. Note that only after the third run of
\LaTeX{} will all references be correct. Unless you change a reference
you do not have to do the \texttt{bibtex} step again.

You can of course use \LaTeX\ rather than pdf\LaTeX, but pdf\LaTeX\ is preferred,
as things like clicking on cross-references and links to publications in the bibliography
work much more reliably with pdf\LaTeX.


%-------------------------------------------------------------------------------
\section{\BibTeX\ database}
%-------------------------------------------------------------------------------

One or more files with the extension \texttt{.bib} 
(in this example: \texttt{atlas\_paper.bib}) should contain all the references. 
The files may also contain references that you do not use, so they may act like a
library of references. 

You should check that you base your \File{.bib} file on the examples provided,
as it has the references in the style recommended for ATLAS publications.
This will definitely save time in the reviewing process!

If you use \Package{biblatex}, you should include commands like:
%
\begin{verbatim}
  \addbibresource{../atlas_latex.bib}
  \addbibresource{atlas_biblatex.bib}
\end{verbatim}
%
in the preamble of our document.

If you use \BibTeX\ you include the source file(s) at the place where the bibliography should be printed as follows:
%
\begin{verbatim}
{\raggedright
  \bibliography{../atlas_latex,%
  atlas_biblatex}
}
\end{verbatim}
at the point where the bibliography should be printed.

If you use \Package{atlaspackage}, then \Package{biblatex} is used with the \BibTeX\ backend by default.
Turn off the use with the option \Option{biblatex=false}.

If you want to use the \Package{biber} backend, include \Package{atlaspackage} with the option \Option{backend=biber} and
use the command \texttt{biber} instead of \texttt{bibtex} to process the \texttt{.bib} files.
Note that \texttt{biber} returns an error if it encounters an empty \texttt{.bib} file.


%-------------------------------------------------------------------------------
\section{ATLAS bibliography style files}
\label{sec:atlasbst}
%-------------------------------------------------------------------------------

The format of the references in your ATLAS paper depends on the journal to which you are submitting,
but in general we can classify the journal styles in two categories: 
those which require the title of the references and those which do not. 
To ensure the homogeneity in all ATLAS publications, 
\Package{biblatex} and \BibTeX style files are provided for each of these categories 
along with an example file that illustrates how different types of bibliographic material should be referenced.

The \Package{biblatex} style file can be found in the directory \File{latex},
while the \BibTeX\ style files can be found in the directory \File{bibtex/bst} of the \Package{atlaslatex} package.

For the final version of ATLAS notes, both internal and public, a style without the title for papers in journals is recommended.
For draft versions, it is good to include the title, as it is then clearer what the reference refers to.


%-------------------------------------------------------------------------------
\subsection{\Package{biblatex} style file}
%-------------------------------------------------------------------------------

If you use \Package{biblatex} a style file: \Package{atlasbiblatex.sty}
which you should include with a normal \Macro{usepackage} command.
By default, this style includes the document title.
The title can be turned off by including the option \Option{articletitle=false}.
This option turns off the title for entry types \texttt{@Article} and \texttt{@Booklet},
as \texttt{@Booklet} should be used as the entry type for CONF and PUB notes.

To print the bibliography include the command:
%
\begin{verbatim}
  \printbibliography
\end{verbatim}
%
where it should get printed.


%-------------------------------------------------------------------------------
\subsection{\BibTeX\ style files}
%-------------------------------------------------------------------------------

If you use \BibTeX, you should choose between the two style files given below, 
depending on the journal to which they wish to submit their paper.
These style files have been successfully tested in the framework provided by 
the journals listed in the following sections and with the standard ATLAS document template.


%-------------------------------------------------------------------------------
\subsubsection{\BibTeX\ style file for journals with the title in the reference}

Journals:
\begin{itemize}\setlength{\parskip}{0pt}\setlength{\itemsep}{0pt}
\item \textbf{JHEP}
\item \textbf{JINST}
\item \textbf{NJP}
\end{itemize}
%
\BibTeX\ style file: \Package{atlasBibStyleWithTitle.bst}

\noindent Include at the end of your \File{.tex} file the following lines:
\begin{verbatim}
\bibliographystyle{bibtex/bst/atlasBibStyleWithTitle}
\bibliography{atlas-bibtex}
\end{verbatim}

You can use the \BibTeX\ style \File{JHEP.bst} for papers that are supposed to be submitted to JHEP.
However, note that JHEP only prints the arXiv entry etc.\ if the entry type is \texttt{@Article}.
In the examples included in this document,
the entry type \texttt{@Booklet} is used for preprints and CONF notes,
as this works best with other \BibTeX\ styles.
If you are planning to submit to JHEP/JINST and use \File{JHEP.bst} 
replace all \texttt{@Booklet} entry types with \texttt{@Article}.


%-------------------------------------------------------------------------------
\subsubsection{\BibTeX\ style file for journals without the title in the reference}

Journals:
\begin{itemize}\setlength{\parskip}{0pt}\setlength{\itemsep}{0pt}
\item \textbf{EPJC}
\item \textbf{NPB}
\item \textbf{PLB}
\item \textbf{PRD}
\item \textbf{PRL}
\end{itemize}
%
\BibTeX\ style file: \Package{atlasBibStyleWoTitle.bst}

\noindent Include at the end of your \File{.tex} file the following lines:
\begin{verbatim}
\bibliographystyle{bibtex/bst/atlasBibStyleWoTitle}
\bibliography{atlas-bibtex}
\end{verbatim}


%-------------------------------------------------------------------------------
\section{Journal names}
%-------------------------------------------------------------------------------

It is often the case that one sees several different abbreviations for journal
names in one set of references.
In order to try to get round this problem, macros are defined that
contain the standard abbreviations.
It is then also possible to modify the abbreviation if a journal uses a different convention from ours.

The abbreviations can be found in the style file \File{latex/atlasjournal.sty},
which is included by default if you load \Package{atlasphysics}.
If you use \BibTeX, you should give the journal name in the form
\verb|journal = "\PLB{}",|. 
If you use \Package{biblatex} the form \verb|journal = "\PLB",| works without problems.
This style file also defines several other abbreviations that can be adjusted to the
journal style. 
Standard sets for different journals can be provided by an option in the future.


%-------------------------------------------------------------------------------
\section{References from Inspire}
\label{sc:inquire}
%-------------------------------------------------------------------------------

A common way to find a reference is using Inspire~\cite{inspire}.
You can select the output format as BibTeX and simply copy the result(s) to your \File{.bib} file.
In order that the reference follows the ATLAS conventions you need to do the following,
assuming that the reference is for an LHC collaboration paper:
\begin{enumerate}
\item Change the field name \texttt{author} to \texttt{xauthor}.
\item Change the field name \texttt{collaboration} to \texttt{author} and write the collaboration in the form\\
  \verb|"{ATLAS Collaboration}"|. Note the use of both quotes and curly braces.
\item Either replace the journal name with the appropriate macro, e.g.\ \verb|"Phys.Lett."| with
  \verb|"\PLB"|; or insert spaces between the parts of the name, i.e.\ \verb|"Phys.\ Lett."|.
  Note the use of \verb|\ | (you can also use \verb|{}|) instead of just a space, 
  as a regular interword space should be inserted and not an end of sentence space.
\item Remove the journal letter from the volume and include it in the journal,
  e.g.\ \enquote{\EPJC}, \enquote{\PRD}. 
\end{enumerate}
If the reference is for another collaboration, rename the \texttt{collaboration} field to
\texttt{xcollaboration} and insert \verb|{NonLHC Collaboration} and| at the beginning of the 
\texttt{author} field.

Instead of renaming the \texttt{author} or \texttt{collaboration} fields, you can of course simply delete them!


%-------------------------------------------------------------------------------
\section{\BibTeX\ tips}
%-------------------------------------------------------------------------------

\begin{itemize}
\item A bibliographic item is created in the \File{.bib} file as:
\begin{verbatim}
@Article{lhcCollaboration:2012,
  author = "...",
  title  = "...",
  further bibliographic information}
}
\end{verbatim}
  The identifier directly after the document type declaration is how one should refer to this item inside the main \File{.tex} file.
  Use a non-breaking space between the citation and the reference, i.e.\
  \verb|... measured previously~\cite{lhcCollaboration:2012}|.
\item When referencing ATLAS CONF notes, the URL to the CDS page should be included.
  For this to work, in the preamble of your \File{.tex} document add
  \texttt{\textbackslash usepackage\{hyperref\}}.
  Note that \Package{hyperref} is included by default if you use \Package{atlasdoc} and/or \Package{atlaspackage}.
\item Depending on the style that is used,
  if the DOI is filled and the \texttt{hyperref} package loaded, 
  the title of the journal can be highlighted in blue and become a hyperlink to the online paper.
\item When referencing papers from journals like PRD, PLB, etc.,
  one has to be careful not to include the \enquote{D} or \enquote{B} as part of the volume.
  Instead these belong to the journal name. 
  You can either use the macros that have been added to the \File{.bst} style files for these journals, or
  the macros defined in \Package{atlasjournal}.
  If you want to use \Package{biblatex} or other \File{bst} files, it is probably better to use the
  \Package{atlasjournal} definitions.
\end{itemize}

In earlier versions of this document, it was recommended to include the \Package{cite} package, 
if you use \BibTeX\ and want to cite multiple references in the format [m-n].
However, the journal style files can do this for you by using the option \Option{sort\&compress} option if \Package{natbib} is used.
The \texttt{revtex} style also does this for you.
If you use \texttt{biblatex} use the option \Option{style=numeric-comp}.

%-------------------------------------------------------------------------------
\section{Examples}
%-------------------------------------------------------------------------------

\begin{itemize}
\item LHC Collaboration~\cite{lhcCollaboration:2012}
\item Other Collaboration~\cite{otherCollaboration:2007}
\item Individual authors~\cite{authors:2008}
\item arXiv only~\cite{arxivOnly:2009}
\item arXiv only submitted to a journal~\cite{arxivSub:2011}
\item ATLAS CONF Note~\cite{atlasConf:2012}
\end{itemize}

While the \texttt{collaboration} field is a nice idea, it is not supported by many \BibTeX\ styles.
Hence in \Ref{\cite{lhcCollaboration:2012}}, \texttt{collaboration} has been renamed to \texttt{author} and
the \texttt{author} field has been renamed as \texttt{xauthor}. If you use \texttt{collaboration} and omit
\texttt{author} you will get a warning when you run \texttt{bibtex}.

Note that in \Ref{\cite{atlasConf:2012}} the entry type \texttt{@Article} used to be used and the field \texttt{journal} 
was abused for the conference note number. This is a result of the \BibTeX\ restrictions on the entry types.
The current version uses \texttt{@Booklet} and includes the CONF note number via the \texttt{howpublished} field.
In general, \texttt{biblatex} provides a lot more entry types.
This is one of the reasons for the  move to \texttt{biblatex} as the default for the ATLAS templates.
All documentation uses \Package{biblatex} with the backend \Package{biber}.
The default for ATLAS documents is to use the backend \Package{bibtex}.

For papers that have been submitted to a journal, but not yet published, use the entry type \texttt{@Article}.
You should specify the journal in the \texttt{journal} field in form:\\
\texttt{journal = "submitted to \Macro{PLB}{}\{\}"} or\\
\texttt{journal = "accepted by \Macro{PLB}{}\{\}"}.


%-------------------------------------------------------------------------------
\section*{History}
%-------------------------------------------------------------------------------

\begin{description}
\item[2013-08-13: Cristina Oropeza Barrera] First version of the document released.
\item[2014-08-14: Ian Brock] Updated the example references a bit and gave a bit more background information.
\item[2014-12-03: Ian Brock] Text taken from paper template and merged into this document. 
  Document adjusted for use of \Package{biblatex} as the default.
\end{description}

%-------------------------------------------------------------------------------
% Print bibliography using biblatex
\printbibliography
%-------------------------------------------------------------------------------
% Old style bibtex bibliography
%\bibliographystyle{../../bibtex/bst/atlasBibStyleWithTitle}
%\bibliography{atlas_bibtex,atlas_biblatex,../atlas_latex}

\end{document}
