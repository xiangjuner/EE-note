%-------------------------------------------------------------------------------
% This file provides a skeleton ATLAS document.
%-------------------------------------------------------------------------------
% \pdfoutput=1
% The \pdfoutput command is needed by arXiv/JHEP/JINST to ensure use of pdflatex.
% It should be included in the first 5 lines of the file.
%-------------------------------------------------------------------------------
% Specify where ATLAS LaTeX style files can be found.
\newcommand*{\ATLASLATEXPATH}{../../latex/}
% Use this variant if the files are in a central location, e.g. $HOME/texmf.
% \newcommand*{\ATLASLATEXPATH}{}
%-------------------------------------------------------------------------------
\documentclass[UKenglish,texlive=2016,txfonts=true]{\ATLASLATEXPATH atlasdoc}
% The language of the document must be set: usually UKenglish or USenglish.
% british and american also work!
% Commonly used options:
%  texlive=YYYY          Specify TeX Live version (2016 is default).
%  atlasstyle=true|false Use ATLAS style for document (default).
%  coverpage             Create ATLAS draft cover page for collaboration circulation.
%                        See atlas-draft-cover.tex for a list of variables that should be defined.
%  cernpreprint          Create front page for a CERN preprint.
%                        See atlas-preprint-cover.tex for a list of variables that should be defined.
%  PAPER                 The document is an ATLAS paper (draft).
%  CONF                  The document is a CONF note (draft).
%  PUB                   The document is a PUB note (draft).
%  BOOK                  The document is of book form, like an LOI or TDR (draft)
%  txfonts=true|false    Use txfonts rather than the default newtx - needed for arXiv submission.
%  paper=a4|letter       Set paper size to A4 (default) or letter.

%-------------------------------------------------------------------------------
% Extra packages:
\usepackage{\ATLASLATEXPATH atlaspackage}
% Commonly used options:
%  biblatex=true|false   Use biblatex (default) or bibtex for the bibliography.
%  backend=biber         Use the biber backend rather than bibtex.
%  subfigure|subfig|subcaption  to use one of these packages for figures in figures.
%  minimal               Minimal set of packages.
%  default               Standard set of packages.
%  full                  Full set of packages.
%-------------------------------------------------------------------------------
% Style file with biblatex options for ATLAS documents.
\usepackage{\ATLASLATEXPATH atlasbiblatex}

% Package for creating list of authors and contributors to the analysis.
\usepackage{\ATLASLATEXPATH atlascontribute}

% Useful macros
\usepackage{\ATLASLATEXPATH atlasphysics}
% See doc/atlas_physics.pdf for a list of the defined symbols.
% Default options are:
%   true:  journal, misc, particle, unit, xref
%   false: BSM, heppparticle, hepprocess, hion, jetetmiss, math, process, other, texmf
% See the package for details on the options.

% Files with references for use with biblatex.
% Note that biber gives an error if it finds empty bib files.
%\addbibresource{MC-snippets.bib}

\addbibresource{../../bib/ATLAS.bib}
\addbibresource{../../bib/ATLAS-errata.bib}
\addbibresource{../../bib/ATLAS-useful.bib}
\addbibresource{../../bib/CMS.bib}
\addbibresource{../../bib/ConfNotes.bib}
\addbibresource{../../bib/PubNotes.bib}

% Paths for figures - do not forget the / at the end of the directory name.
\graphicspath{{../../logos/}{figures/}}

% Add you own definitions here (file MC-snippets-defs.sty).
\usepackage{MC-snippets-defs}

%-------------------------------------------------------------------------------
% Generic document information
%-------------------------------------------------------------------------------

% Title, abstract and document 
\input{MC-snippets-metadata}
% Author and title for the PDF file
\hypersetup{pdftitle={ATLAS document},pdfauthor={The ATLAS Collaboration}}

%-------------------------------------------------------------------------------
% Content
%-------------------------------------------------------------------------------
\begin{document}

\maketitle

\tableofcontents

% List of contributors - print here or after the Bibliography.
%\PrintAtlasContribute{0.30}
%\clearpage

%-------------------------------------------------------------------------------
% Intro
\section{Introduction}

This document is a collection of short descriptions of the baseline Standard Model processes
produced as part of the ATLAS MC16 production campaign. Often a short and a long description 
is provided, depending on whether a sample is used as a background or a signal sample in an
analysis, respectively.


It is assumed that paper editors will make a final pass through the wording, e.g.\ to avoid
acronyms being introduced multiple times.
The descriptions contain the appropriate citations which are included by default in 
the \texttt{atlaslatex} package as well.
These citations often reflect decades of theory work and would have typically been agreed upon 
with the generator developers, who rely on them to secure funding for future generator development.
PMG therefore strongly encourages \emph{keeping all recommended citations} for any given snippet.


Please note that the generator versions can generally change from sample to sample. 
A change in the third digit typically indicates some sort of technical bug fix that
does not affect the physics modelling otherwise. 
In order to save CPU time, samples are often regenerated only when they are affected 
by a (sufficiently severe) bug and so even within a set of final states 
of any given process, the generator version may differ. 


%-------------------------------------------------------------------------------
% Common generator details
%\include{MC}
%-------------------------------------------------------------------------------
% Minbias
\include{Pileup}
%-------------------------------------------------------------------------------
% V+jets
\include{BJ}
%-------------------------------------------------------------------------------
% Multiboson
\include{MB}
%-------------------------------------------------------------------------------
% Higgs processes                                                                                             
\include{Higgs}
%-------------------------------------------------------------------------------
% Top
\include{TQ}
%-------------------------------------------------------------------------------
% RareTop
\section{Rare top-quark processes}
%\label{sec:RT}

This section describes the samples used for rare top-quark processes.
Section~\ref{subsec:ttH} describes the \ttH\ samples.
Section~\ref{subsec:ttV} describes the \ttV\ ($V=W/Z$) samples.
Section~\ref{subsec:ttgamma} describes the \ttgamma\ samples.
Section~\ref{subsec:tHq} describes the \tH\ samples.
Section~\ref{subsec:tZq} describes the \tZq\ samples.
Section~\ref{subsec:tWZ} describes the \tWZ\ samples.
%\tgamma\ samples are described in Section~\ref{subsec:tgamma}.
Finally, Section~\ref{subsec:tttt} describes the \tttt\ samples.

\input{ttH}
\input{ttV}
\input{ttgamma}
%%%%%%%%%%%%%%%%%%%%%%%%%%%%%%%%%%%%%%%%%%%
%%%              tH                    %%%
%%%%%%%%%%%%%%%%%%%%%%%%%%%%%%%%%%%%%%%%%%%
\subsection[\tHq]{\tHq}
\label{subsec:tHq}

This section describes the MC samples used for the modelling of \tH\ production.
Section~\ref{subsubsec:tHq_aMCP8} describes the \MGNLOPY[8] samples,

\subsubsection[MadGraph5\_aMC@NLO+Pythia8]{\MGNLOPY[8]}
\label{subsubsec:tHq_aMCP8}

\paragraph{Samples}
%\label{par:tHq_aMCP8_samples}

The descriptions below correspond to the samples in Table~\ref{tab:tHq_aMCP8}.

\begin{table}[htbp]
\begin{center}
\caption{Nominal \tH\ samples produced with \MGNLOPY[8].} 
\label{tab:tHq_aMCP8}
\begin{tabular}{ l | l }
\hline
DSID range & Description \\
\hline
346188 & \tHq\, \hgg\, four flavour \\
346221 & \tHq\, \hgg\, five flavour \\
346229 & \tHq\, \hbb\, four flavour \\
346230 & \tHq\, \htautau/\hzz/\hww\, four flavour \\
\hline
\end{tabular}
\end{center}
\end{table}


\paragraph{Short description:}

The production of \tHq events was modelled using the \MGNLO[2.3.3]~\cite{Alwall:2014hca}
generator at NLO with the \NNPDF[3.0nlo]~\cite{Ball:2014uwa} parton distribution function~(PDF).
The events were interfaced with \PYTHIA[8.230]~\cite{Sjostrand:2014zea} using the A14 tune~\cite{ATL-PHYS-PUB-2014-021} and the
\NNPDF[2.3lo]~\cite{Ball:2014uwa} PDF set.
The decays of bottom and charm hadrons were simulated using the \EVTGEN program~\cite{Lange:2001uf}. 

\paragraph{Long description:}

The \tHq samples were simulated using the \MGNLO[2.3.3]~\cite{Alwall:2014hca}
generator at NLO with the \NNPDF[3.0nlo]~\cite{Ball:2014uwa} parton distribution function~(PDF). The events were interfaced with
\PYTHIA[8.230]~\cite{Sjostrand:2014zea}~ using the A14 tune~\cite{ATL-PHYS-PUB-2014-021} and the \NNPDF[2.3lo]~\cite{Ball:2014uwa} PDF set. 
The top quark was decayed at LO using \MADSPIN~\cite{Frixione:2007zp,Artoisenet:2012st} to preserve spin correlations,
whereas the Higgs boson was decayed by \PYTHIA in the parton shower.
Both the four-flavour and five-flavour schemes were considered.
The functional form of the renormalisation and factorisation scales was set to the 
default scale $0.5\times \sum_i \sqrt{m^2_i+p^2_{\text{T},i}}$, where the sum runs over 
all the particles generated from the matrix element calculation.
The decays of bottom and charm hadrons were simulated using the \EVTGEN program~\cite{Lange:2001uf}.

\input{tZq}
\input{tWZ}
%\input{tgamma}
\input{tttt}

%-------------------------------------------------------------------------------                         
% Jets
\include{MJ}
%-------------------------------------------------------------------------------
% Photon processes                                                                                             
\section{Photon processes}
%\label{sec:GJ}

The following paragraphs describe the set-up of the current ATLAS $\gamma$+jets and $\gamma\gamma$+jets baseline samples. 

\subsection[Sherpa (MEPS@NLO)]{\SHERPA (\MEPSatNLO)}
%\label{sec:gammajets-sherpa-nlo}

\subsubsection*{Samples}
%\label{sec:gammajets-sherpa-nlo-samples}

The descriptions below correspond to the samples in Table~\ref{tab:gammajets-sherpa-nlo}.
\begin{table}[!htbp]
\begin{center}
\caption{$\gamma$+jets and $\gamma\gamma$+jets samples with \SHERPA NLO} \label{tab:gammajets-sherpa-nlo}
\begin{tabular}{ l | l }
\hline
DSID range & Description \\
\hline
364541--364547 &  Single photon \\ 
364350--364354 &  Diphoton  \\
\hline
\end{tabular}
\end{center}
\end{table}



%same description as in~\ref{sec:vjets-sherpa-vjets}
\subsubsection{$\gamma$+jets}
%\label{sec:gammajets-sherpa-nlo-singlephoton}


\paragraph{Short description:}

Prompt single-photon production was simulated with the
\SHERPA[2.2]~\cite{Bothmann:2019yzt} generator. In this set-up, NLO-accurate
matrix elements for up to two partons, and LO-accurate matrix elements for up
to four partons were calculated with the Comix~\cite{Gleisberg:2008fv} and
\OPENLOOPS~\cite{Buccioni:2019sur,Cascioli:2011va,Denner:2016kdg} libraries. They were matched
with the \SHERPA parton shower~\cite{Schumann:2007mg} using the \MEPSatNLO
prescription~\cite{Hoeche:2011fd,Hoeche:2012yf,Catani:2001cc,Hoeche:2009rj}
with a dynamic merging cut~\cite{Siegert:2016bre} of 20~\GeV.
Photons were required to be isolated according to
a smooth-cone isolation criterion~\cite{Frixione:1998jh}. Samples were generated using the
\NNPDF[3.0nnlo] PDF set~\cite{Ball:2014uwa}, along with the dedicated set of
tuned parton-shower parameters developed by the \SHERPA authors.

\paragraph{Long description:}

Prompt single-photon production was simulated with the
\SHERPA[2.2]~\cite{Bothmann:2019yzt} parton shower Monte Carlo
generator. In this set-up, NLO-accurate
matrix elements for up to two partons, and LO-accurate matrix elements for up
to four partons were calculated with the Comix~\cite{Gleisberg:2008fv} and
\OPENLOOPS~\cite{Buccioni:2019sur,Cascioli:2011va,Denner:2016kdg} libraries.
The default \SHERPA parton shower~\cite{Schumann:2007mg} based on
Catani--Seymour dipole factorisation and the cluster hadronisation model~\cite{Winter:2003tt}
were used. They employed the dedicated set of tuned parameters developed by the
\SHERPA authors for this generator version and the \NNPDF[3.0nnlo] PDF
set~\cite{Ball:2014uwa}.

The NLO matrix elements for a given jet multiplicity were matched to the parton
shower using a colour-exact variant of the MC@NLO
algorithm~\cite{Hoeche:2011fd}. Different jet multiplicities were then merged
into an inclusive sample using an improved CKKW matching
procedure~\cite{Catani:2001cc,Hoeche:2009rj} which was extended to NLO
accuracy using the \MEPSatNLO prescription~\cite{Hoeche:2012yf}.
The merging cut was set dynamically at a scale of 20~\GeV{}
according to the prescription in Ref.~\cite{Siegert:2016bre}.

The renormalisation and factorisation scales for the photon-plus-jet core
process were set to the transverse energy of the photon, $E_\text{T}^\gamma$.
The strong coupling constant was set to $\alphas (m_Z)= 0.118$ and the QED coupling
constant was evaluated in the Thomson limit. Photons from the matrix elements were required 
to be central, by being within the rapidity range $|y_{\gamma}|<2.7$, and isolated according to a 
smooth-cone isolation criterion~\cite{Frixione:1998jh} with $\delta_0=0.1$, $\epsilon_{\gamma}=0.1$ and $n=2$.

The effects of QCD scale uncertainties were evaluated~\cite{Bothmann:2016nao} using
seven-point variations of the factorisation and renormalisation scales in the matrix elements.
The scales were varied independently by factors of $0.5$ and $2$, avoiding variations in opposite directions.

PDF uncertainties for the nominal PDF set were
evaluated using the 100 variation replicas, as well as $\pm 0.001$ shifts
of \alphas. Additionally, the results were cross-checked using the central values of the 
\CT[14nnlo]~\cite{Dulat:2015mca} and \MMHT[nnlo]~\cite{Harland-Lang:2014zoa} 
PDF sets.

%\textcolor{red}{ Several Monte Carlo slices were generated to ensure good statistics over the whole phase space. In each of them, the photon was required to fall within a $E_{T}^{\gamma}$ range that depends on the slice avoiding any double-counting when the different slices are combined.}



\subsubsection[yy+jets]{ $\gamma\gamma$+jets}
%\label{sec:gammajets-sherpa-nlo-diphoton}

\paragraph{Short description:}

Prompt diphoton production was simulated with the
\SHERPA[2.2]~\cite{Bothmann:2019yzt} generator. In this set-up, NLO-accurate
matrix elements for up to one parton, and LO-accurate matrix elements for up
to three partons were calculated with the Comix~\cite{Gleisberg:2008fv} and
\OPENLOOPS~\cite{Buccioni:2019sur,Cascioli:2011va,Denner:2016kdg} libraries. They were matched
with the \SHERPA parton shower~\cite{Schumann:2007mg} using the \MEPSatNLO
prescription~\cite{Hoeche:2011fd,Hoeche:2012yf,Catani:2001cc,Hoeche:2009rj}
with a dynamic merging cut~\cite{Siegert:2016bre} of 10~\GeV.
Photons were required to be isolated according to a smooth-cone isolation
criterion~\cite{Frixione:1998jh}. Samples were generated using the
\NNPDF[3.0nnlo] PDF set~\cite{Ball:2014uwa}, along with the dedicated set of tuned
parton-shower parameters developed by the \SHERPA authors.



\paragraph{Long description:}

Prompt diphoton production was simulated with the
\SHERPA[2.2]~\cite{Bothmann:2019yzt} parton shower Monte Carlo
generator. In this set-up, NLO and LO-accurate
matrix elements were calculated with the Comix~\cite{Gleisberg:2008fv} and
\OPENLOOPS~\cite{Buccioni:2019sur,Cascioli:2011va,Denner:2016kdg} libraries.
The default \SHERPA parton shower~\cite{Schumann:2007mg} based on
Catani--Seymour dipole factorisation and the cluster hadronisation model~\cite{Winter:2003tt}
were used. They employed the dedicated set of tuned parameters developed by the
\SHERPA authors for this generator version and the \NNPDF[3.0nnlo]
PDF set~\cite{Ball:2014uwa}.

The NLO matrix elements for a given jet multiplicity were matched to the parton
shower using a colour-exact variant of the MC@NLO algorithm~\cite{Hoeche:2011fd}. 
Different jet multiplicities were then merged
into an inclusive sample using an improved CKKW matching
procedure~\cite{Catani:2001cc,Hoeche:2009rj} which was extended to NLO
accuracy using the \MEPSatNLO prescription~\cite{Hoeche:2012yf}.
The merging cut was set dynamically to a scale of 20~\GeV{},
according to the prescription in Ref.~\cite{Siegert:2016bre}.

The renormalisation and factorisation scales for the diphoton core process
were set to the invariant mass of the photon pair, $m_{\gamma\gamma}$.
The strong coupling constant was set to $\alphas (m_Z)= 0.118$ and the QED coupling
constant was evaluated in the Thomson limit. Photons from the matrix elements were required to be central, 
by being within the rapidity range $|y_{\gamma}|<2.7$, and isolated according to a smooth-cone isolation 
criterion~\cite{Frixione:1998jh} with $\delta_0=0.1$, $\epsilon_{\gamma}=0.1$ and $n=2$. 
Additionally, the photons were required to be separated by $\Delta R(\gamma_1,\gamma_2) > 0.2$.


The effects of QCD scale uncertainties were evaluated~\cite{Bothmann:2016nao} using
seven-point variations of the factorisation and renormalisation scales in the matrix elements.
The scales were varied independently by factors of $0.5$ and $2$, avoiding variations in opposite directions.

PDF uncertainties for the nominal PDF set were
evaluated using the 100 variation replicas, as well as $\pm 0.001$ shifts
of \alphas. Additionally, the results were cross-checked using the central values of the 
\CT[14nnlo]~\cite{Dulat:2015mca} and \MMHT[nnlo]~\cite{Harland-Lang:2014zoa} 
PDF sets.
%\textcolor{red}{ Several Monte Carlo slices were generated to ensure good statistics over the whole phase space. In each of them, the photon pair was required to fall within a $m_{\gamma\gamma}$ range that depends on the slice avoiding any double-counting when the different slices are combined.}




\subsection[Sherpa (MEPS@NLO)]{\SHERPA (\MEPSatNLO)}
%\label{sec:gammajets-sherpa-lo}


\subsubsection*{Samples}
%\label{sec:gammajets-sherpa-lo-samples}

The descriptions below correspond to the samples in Table~\ref{tab:gammajets-sherpa-lo}.
\begin{table}[!htbp]
\begin{center}
\caption{$\gamma$+jets and  $\gamma\gamma$+jets samples with \SHERPA LO} \label{tab:gammajets-sherpa-lo}
\begin{tabular}{ l | l }
\hline
DSID range & Description \\
\hline
361039--361062 &  Single photon \\
303727--303742 &  Diphoton \\
\hline
\end{tabular}
\end{center}
\end{table}

\subsubsection[y+jets]{$\gamma$+jets}
%\label{sec:gammajets-sherpa-lo-singlephoton}


\paragraph{Description:}

Prompt single-photon production was simulated using the \SHERPA[2.1]~\cite{Bothmann:2019yzt}
generator. The tree-level matrix elements, generated for up to three
additional partons, were merged with the initial- and final-state parton showers using the
\MEPSatLO prescription~\cite{Hoeche:2009rj}. The \CT[10nlo] set of PDFs~\cite{Lai:2010vv} was
used to parameterise the proton structure in conjunction with the dedicated set of tuned
parton-shower parameters developed by the \SHERPA authors for this generator version. A
modified version of the cluster model~\cite{Winter:2003tt} was used
for the description of the fragmentation into hadrons. Photons from the matrix elements were
required to be isolated according to a smooth-cone hadronic isolation criterion~\cite{Frixione:1998jh}
with $\delta_0=0.3$, $\epsilon_{\gamma}=0.025$ and $n=2$.




\subsubsection[yy+jets]{ $\gamma\gamma$+jets}
%\label{sec:gammajets-sherpa-lo-diphoton}


\paragraph{Description:}

Prompt diphoton production was simulated using the \SHERPA[2.1]~\cite{Bothmann:2019yzt}
generator. The tree-level matrix elements, generated for up to two
additional partons, were merged with the initial- and final-state parton showers using the
\MEPSatLO prescription~\cite{Hoeche:2009rj}. The \CT[10nlo] set of PDFs~\cite{Lai:2010vv} was
used to parameterise the proton structure in conjunction with the dedicated set of tuned
parton-shower parameters developed by the \SHERPA authors for this generator version. A
modified version of the cluster model~\cite{Winter:2003tt} was used
for the description of the fragmentation into hadrons. Photons from the matrix elements were
required to be isolated according to a smooth-cone hadronic isolation criterion~\cite{Frixione:1998jh}
with $\delta_0=0.3$, $\epsilon_{\gamma}=0.025$ and $n=2$. Additionally, the photons were 
required to be separated by $\Delta R(\gamma_1,\gamma_2) > 0.2$.

\subsection[Pythia (LO)]{\PYTHIA (LO)}
%\label{sec:gammajets-pythia-lo}

The descriptions below correspond to the samples in Table~\ref{tab:gammajets-pythia-lo}.
\begin{table}[!htbp]
\begin{center}
\caption{$\gamma$+jets and  $\gamma\gamma$+jets samples with \PYTHIA}
\label{tab:gammajets-pythia-lo}
\begin{tabular}{ l | l }
\hline
DSID range & Description \\
\hline
423099--423112 &  Single photon \\
344008, 302520--34, 364423 &  Diphoton \\
\hline
\end{tabular}
\end{center}
\end{table}



\subsubsection[y+jets]{$\gamma$+jets}
%\label{sec:gammajets-pythia-lo-singlephoton}

\paragraph{Description:}

Prompt single-photon production was simulated using the \PYTHIA[8.186]~\cite{Sjostrand:2007gs} generator. 
Events were generated using tree-level matrix elements for photon-plus-jet final
states as well as LO QCD dijet events, with the inclusion of initial-
and final-state parton showers. The fragmentation component was
modelled by final-state QED radiation arising from calculations of all
$2\rightarrow 2$ QCD processes. The \NNPDF[2.3lo]~\cite{Ball:2012cx} PDF
set was used in the matrix element calculation, the parton shower, and
the simulation of the multi-parton interactions. The samples
include a simulation of the underlying event with parameters set
according to the A14 tune~\cite{ATL-PHYS-PUB-2014-021}. The Lund
string model~\cite{Andersson:1983ia,Sjostrand:1984ic} was used for the
description of the fragmentation into hadrons.




\subsubsection[yy+jets]{ $\gamma\gamma$+jets}
%\label{sec:gammajets-pythia-lo-diphoton}

\paragraph{Description:}

Prompt diphoton production was simulated using the 
\PYTHIA[8.186]~\cite{Sjostrand:2007gs} generator. Events were 
generated using tree-level matrix elements for diphoton final states,
with the inclusion of initial- and final-state parton showers.  The
fragmentation component was modelled by final-state QED radiation
arising from calculations of photon-plus-jet processes in dedicated
samples. The \NNPDF[2.3lo]~\cite{Ball:2012cx} PDF set was used in the
matrix element calculation, the parton shower, and in the simulation of the
multi-parton interactions. The samples include a simulation of the
underlying event with parameters set according to the A14
tune~\cite{ATL-PHYS-PUB-2014-021}. The Lund string
model~\cite{Andersson:1983ia,Sjostrand:1984ic} was used for the
description of the fragmentation into hadrons.




%-------------------------------------------------------------------------------


%-------------------------------------------------------------------------------
%\section*{Acknowledgements}
%-------------------------------------------------------------------------------

%% Acknowledgements for papers with collision data
% Version 17-May-2021

% Standard acknowledgements start here
%----------------------------------------------

We thank CERN for the very successful operation of the LHC, as well as the
support staff from our institutions without whom ATLAS could not be
operated efficiently.

We acknowledge the support of ANPCyT, Argentina; YerPhI, Armenia; ARC, Australia; BMWFW and FWF, Austria; ANAS, Azerbaijan; SSTC, Belarus; CNPq and FAPESP, Brazil; NSERC, NRC and CFI, Canada; CERN; ANID, Chile; CAS, MOST and NSFC, China; Minciencias, Colombia; MSMT CR, MPO CR and VSC CR, Czech Republic; DNRF and DNSRC, Denmark; IN2P3-CNRS and CEA-DRF/IRFU, France; SRNSFG, Georgia; BMBF, HGF and MPG, Germany; GSRT, Greece; RGC and Hong Kong SAR, China; ISF and Benoziyo Center, Israel; INFN, Italy; MEXT and JSPS, Japan; CNRST, Morocco; NWO, Netherlands; RCN, Norway; MNiSW and NCN, Poland; FCT, Portugal; MNE/IFA, Romania; JINR; MES of Russia and NRC KI, Russian Federation; MESTD, Serbia; MSSR, Slovakia; ARRS and MIZ\v{S}, Slovenia; DST/NRF, South Africa; MICINN, Spain; SRC and Wallenberg Foundation, Sweden; SERI, SNSF and Cantons of Bern and Geneva, Switzerland; MOST, Taiwan; TAEK, Turkey; STFC, United Kingdom; DOE and NSF, United States of America. In addition, individual groups and members have received support from BCKDF, CANARIE, Compute Canada, CRC and IVADO, Canada; Beijing Municipal Science \& Technology Commission, China; COST, ERC, ERDF, Horizon 2020 and Marie Sk{\l}odowska-Curie Actions, European Union; Investissements d'Avenir Labex, Investissements d'Avenir Idex and ANR, France; DFG and AvH Foundation, Germany; Herakleitos, Thales and Aristeia programmes co-financed by EU-ESF and the Greek NSRF, Greece; BSF-NSF and GIF, Israel; La Caixa Banking Foundation, CERCA Programme Generalitat de Catalunya and PROMETEO and GenT Programmes Generalitat Valenciana, Spain; G\"{o}ran Gustafssons Stiftelse, Sweden; The Royal Society and Leverhulme Trust, United Kingdom.

The crucial computing support from all WLCG partners is acknowledged gratefully, in particular from CERN, the ATLAS Tier-1 facilities at TRIUMF (Canada), NDGF (Denmark, Norway, Sweden), CC-IN2P3 (France), KIT/GridKA (Germany), INFN-CNAF (Italy), NL-T1 (Netherlands), PIC (Spain), ASGC (Taiwan), RAL (UK) and BNL (USA), the Tier-2 facilities worldwide and large non-WLCG resource providers. Major contributors of computing resources are listed in Ref.~\cite{ATL-SOFT-PUB-2020-001}.

%----------------------------------------------
% Created with Glance <Atlas.Glance@cern.ch>


%The \texttt{atlaslatex} package contains the acknowledgements that were valid 
%at the time of the release you are using.
%These can be found in the \texttt{acknowledgements} subdirectory.
%When your ATLAS paper or PUB/CONF note is ready to be published,
%download the latest set of acknowledgements from:\\
%\url{https://twiki.cern.ch/twiki/bin/view/AtlasProtected/PubComAcknowledgements}

%The supporting notes for the analysis should also contain a list of contributors.
%This information should usually be included in \texttt{mydocument-metadata.tex}.
%The list should be printed either here or before the table of contents.


%-------------------------------------------------------------------------------
%\clearpage
%\appendix
%\part*{Appendix}
%\addcontentsline{toc}{part}{Appendix}
%-------------------------------------------------------------------------------

%In a paper, an appendix is used for technical details that would otherwise disturb the flow of the paper.
%Such an appendix should be printed before the Bibliography.


%-------------------------------------------------------------------------------
% If you use biblatex and either biber or bibtex to process the bibliography
% just say \printbibliography here
\printbibliography
% If you want to use the traditional BibTeX you need to use the syntax below.
%\bibliographystyle{bibtex/bst/atlasBibStyleWoTitle}
%\bibliography{MC-snippets,bibtex/bib/ATLAS}
%-------------------------------------------------------------------------------

\end{document}
