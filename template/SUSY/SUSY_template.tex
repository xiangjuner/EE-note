%-------------------------------------------------------------------------------
% This file provides template SUSY group object descriptions and cuts.
\pdfinclusioncopyfonts=1
% This command may be needed in order to get \ell in PDF plots to appear. Found in
% https://tex.stackexchange.com/questions/322010/pdflatex-glyph-undefined-symbols-disappear-from-included-pdf
%-------------------------------------------------------------------------------
%-------------------------------------------------------------------------------
% Specify where ATLAS LaTeX style files can be found.
\makeatletter
\def\input@path{{../../latex/}}
\makeatother

%-------------------------------------------------------------------------------
\documentclass[NOTE, REPORT=false, atlasdraft=false, USenglish]{atlasdoc}
% The language of the document must be set: usually UKenglish or USenglish.
% british and american also work!
% Commonly used options:
%  atlasdraft=true|false This document is an ATLAS draft.
%  texlive=YYYY          Specify TeX Live version (2020 is default).
%  txfonts=true|false    Use txfonts rather than the default newtx
%  paper=a4|letter       Set paper size to A4 (default) or letter.

%-------------------------------------------------------------------------------
% Extra packages:
\usepackage[biblatex=false]{atlaspackage}
% Commonly used options:
%  subfigure|subfig|subcaption  to use one of these packages for figures in figures.
%-------------------------------------------------------------------------------
\usepackage{multirow}

% Useful macros
\usepackage[jetetmiss]{atlasphysics}
% See doc/atlas_physics.pdf for a list of the defined symbols.
% Default options are:
%   true:  journal, misc, particle, unit, xref
%   false: BSM, heppparticle, hepprocess, hion, jetetmiss, math, process,
%          other, snippets, texmf
% See the package for details on the options.

% Package for creating list of authors and contributors to the analysis.
\usepackage{atlascontribute}

% Add you own definitions here (file atlas-document-defs.sty).
% \usepackage{atlas-document-defs}

% Paths for figures - do not forget the / at the end of the directory name.
\graphicspath{{../../logos/}{figures/}}

%-------------------------------------------------------------------------------
% Generic document information
%-------------------------------------------------------------------------------

\AtlasTitle{SUSY group text snippets for INT notes}

\author{ATLAS SUSY Group}

\AtlasRefCode{Version \ATPackageVersion}

\AtlasAbstract{%
  This note contains text snippets and tables that should be included in supporting notes
  from the SUSY group.

  The templates are in American English.
  If wanted, some adaption to British English could be made. 

  % \emph{2019-02-04: This file is a work in progress (WIP) and will probably be updated.
  % Backwards incompatible changes may be made as the examples develop.}
}
% Author and title for the PDF file
\hypersetup{pdftitle={ATLAS SUSY supporting note},pdfauthor={ATLAS SUSY group}}

%-------------------------------------------------------------------------------
% Main document
%-------------------------------------------------------------------------------
\begin{document}

\maketitle

\tableofcontents

%-------------------------------------------------------------------------------
% The executive summary template is provided by the group
\include{executive_summary}
%-------------------------------------------------------------------------------

Please feel free to also include a change log with major updates either before or after the executive summary.

%-------------------------------------------------------------------------------
% The executive summary template is provided by the group
\section{Introduction}
\label{sec:introduction}
%-------------------------------------------------------------------------------

Place a short introduction here.  It is useful to introduce your analysis target signals, place them in context,
and describe any previous analyses that this analysis follows on (particularly if significant pieces are in common).

Please note that in an internal note there is no need for a description of the ATLAS detector,
unless the search uses some unusual or less well-known features of the detector of which reviewers will need a reminder.

%-------------------------------------------------------------------------------
\section{Signal Models}
\label{sec:signals}
%-------------------------------------------------------------------------------

Place the description of your target signal models here, potentially including a description of the ``signal grids'' that you will ultimately use for interpretataion.

%-------------------------------------------------------------------------------
\section{Data and Simulated Event Samples}
\label{sec:samples}
%-------------------------------------------------------------------------------

Place the description of your dataset (including GRL) and Monte Carlo simulated samples here.
Please place any long tables (e.g. lists of used datasets) into appendices.

%-------------------------------------------------------------------------------
\section{Object definition}
\label{sec:objects}
%-------------------------------------------------------------------------------

Please describe your object definition here.
% Object definition recommendations are here: https://twiki.cern.ch/twiki/bin/view/AtlasProtected/SusyObjectDefinitionsr2113TeV

%-------------------------------------------------------------------------------
\section{Event selection}
\label{sec:selection}
%-------------------------------------------------------------------------------

Place the description of your event selection here.  This can include signal region optimization, control region selection, and pre-selection.

%-------------------------------------------------------------------------------
\section{Background Estimation}
\label{sec:backgrounds}
%-------------------------------------------------------------------------------

Place the description of your background estimation here.

%-------------------------------------------------------------------------------
\section{Systematic Uncertainties}
\label{sec:systematics}
%-------------------------------------------------------------------------------

Place the description of your systematic uncertainties here.
% Useful recommendations are maintained by the SUSY group: https://twiki.cern.ch/twiki/bin/view/AtlasProtected/SUSYSpecificRecommendations
% and by the PMG: https://twiki.cern.ch/twiki/bin/view/AtlasProtected/PhysicsModellingGroup#Systematic_uncertainties_and_rew

%-------------------------------------------------------------------------------
\section{Results}
\label{sec:result}
%-------------------------------------------------------------------------------

Place your results here.

% The required information is available here: https://twiki.cern.ch/twiki/bin/viewauth/AtlasProtected/SUSYResultsPresentation
% All figures and tables should appear before the summary and conclusion.
% The package placeins provides the macro \FloatBarrier to achieve this.
% \FloatBarrier


%-------------------------------------------------------------------------------
\section{Conclusion}
\label{sec:conclusion}
%-------------------------------------------------------------------------------

Place your conclusion here.


%-------------------------------------------------------------------------------
% If you use biblatex and either biber or bibtex to process the bibliography
% just say \printbibliography here
% \printbibliography
% If you want to use the traditional BibTeX you need to use the syntax below.
% \bibliographystyle{obsolete/bst/atlasBibStyleWithTitle}
% \bibliography{atlas-document,bib/ATLAS,bib/CMS,bib/ConfNotes,bib/PubNotes}
%-------------------------------------------------------------------------------

%-------------------------------------------------------------------------------
% Print the list of contributors to the analysis
% The argument gives the fraction of the text width used for the names
%-------------------------------------------------------------------------------
\clearpage
The supporting notes for the analysis should also contain a list of contributors.
This information should usually be included in \texttt{mydocument-metadata.tex}.
The list should be printed either here or before the Table of Contents.
\PrintAtlasContribute{0.30}


%-------------------------------------------------------------------------------
\clearpage
\appendix
\part*{Appendices}
\addcontentsline{toc}{part}{Appendices}
%-------------------------------------------------------------------------------

In an ATLAS note, use the appendices to include all the technical details of your work
that are relevant for the ATLAS Collaboration only (e.g.\ dataset details, software release used).
This information should be printed after the Bibliography.

\end{document}
