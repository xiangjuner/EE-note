%-------------------------------------------------------------------------------
% This file provides template EXOT group object descriptions and cuts.
% \pdfinclusioncopyfonts=1
% This command may be needed in order to get \ell in PDF plots to appear. Found in
% https://tex.stackexchange.com/questions/322010/pdflatex-glyph-undefined-symbols-disappear-from-included-pdf
%-------------------------------------------------------------------------------
% Specify where ATLAS LaTeX style files can be found.
\newcommand*{\ATLASLATEXPATH}{../../latex/}
% Use this variant if the files are in a central location, e.g. $HOME/texmf.
% \newcommand*{\ATLASLATEXPATH}{}
%-------------------------------------------------------------------------------
\documentclass[NOTE, atlasdraft=true, texlive=2016, USenglish]{\ATLASLATEXPATH atlasdoc}
% The language of the document must be set: usually UKenglish or USenglish.
% british and american also work!
% Commonly used options:
%  atlasdraft=true|false This document is an ATLAS draft.
%  texlive=YYYY          Specify TeX Live version (2016 is default).
%  txfonts=true|false    Use txfonts rather than the default newtx
%  paper=a4|letter       Set paper size to A4 (default) or letter.

%-------------------------------------------------------------------------------
% Extra packages:
\usepackage{\ATLASLATEXPATH atlaspackage}
% Commonly used options:
%  subfigure|subfig|subcaption  to use one of these packages for figures in figures.
%-------------------------------------------------------------------------------
\usepackage{multirow}

% Useful macros
\usepackage{\ATLASLATEXPATH atlasphysics}
% See doc/atlas_physics.pdf for a list of the defined symbols.
% Default options are:
%   true:  journal, misc, particle, unit, xref
%   false: BSM, heppparticle, hepprocess, hion, jetetmiss, math, process,
%          other, snippets, texmf
% See the package for details on the options.

% Add you own definitions here (file atlas-document-defs.sty).
% \usepackage{atlas-document-defs}

% Paths for figures - do not forget the / at the end of the directory name.
\graphicspath{{\ATLASLATEXPATH ../logos/}{figures/}}

%-------------------------------------------------------------------------------
% Generic document information
%-------------------------------------------------------------------------------

\AtlasTitle{EXOT group text snippets for INT notes}
\AtlasVersion{0.1}
\author{ATLAS EXOT Group}
\AtlasRefCode{EXOT-2018-XX}
\AtlasAbstract{%
  This note contains text snippets and tables that shoudl be included in supporting notes
  from the EXOT group.

  The templates are in American English. Some adaption toBritish English is therefore required. 

  \emph{2018-10-22: This file is very much a work in progress (WIP) and is expected to be updated regularly.
  Bachwards incompatible changes may be made as the examples develop.}
}
% Author and title for the PDF file
\hypersetup{pdftitle={ATLAS EXOT supporting note},pdfauthor={ATLAS EXOT group}}

%-------------------------------------------------------------------------------
% Main document
%-------------------------------------------------------------------------------
\begin{document}

\maketitle

\tableofcontents

\section{Executive Summary}

This section, ideally 2-pages (max), should be placed at the beginning of the internal note.
It should give a high-level overview of the analysis including (but not limited to):
\begin{itemize}
\item physics target and the general characteristics of the signal;
\item analysis strategy;
\item general characteristics of the control, validation, and signal regions;
\item background estimation strategy overview;
\item highlight major or most important points of the analysis;
\item team overview task list;
\item list of all critical tasks, who is responsible for each, and what else they are working on outside of this analysis.
\end{itemize}
split as in the subsections below.

\subsection{Target} 

\(\mathcal{O}\)(1 paragraph)
Is this a new analysis? If not, what are the main improvements expected with respect to the previous version?
What is the target publication date / conference?

\subsection{Context} 

Motivate this analysis in 1 paragraph: why is this signature interesting? Which kind of models are you probing?

How is the analysis done is 1 paragraph: what are the main BG processes and how do you estimate them (are they MC- or data-driven,
what is the general idea of the control regions, \ldots), general characteristics of the PL fit (which distribution, binned?, \ldots)

\subsection{Milestones}

 Table giving a factual list of who is working on what and what else they do; the idea is to show how the team can / does progress. 

%Example : 
The following table summarizes the tasks to be worked on by analysis team.
This is not a complete analysis outline but only an overview of the further steps to be taken as of the time of writing.
Details are not provided here but in the dedicated sections throughout this note.
Tasks which are based on established techniques and straightforward to achieve are marked green in the table.
Tasks which require new work are marked red.
Concerning the involved people, the responsible student supervisors and analysis coordinators are already mentioned in the list of contributions above,
which shall not be repeated here.
A fair overview of all single tasks including past work and of all relevant team members is only given in the list of contributions above!
It is also worth noting that some of the tasks listed below are being worked on in parallel. 

\begin{table}[ht]
  \caption{Milestones in the analysis.}%
  \label{tab:Miles_Ahead} 
  % \resizebox{\textwidth}{!}{
\begin{tabular}{llll} 
  \toprule
  \textbf{Task} & \textbf{Analyzer} & \textbf{Role} & \textbf{Other responsibilities} \\
  \midrule
  \multicolumn{4}{p{\textwidth}}{\textbf{Describe a first milestone.}} \\
  \midrule
  \textcolor{green}{A straightforward task}       & Name         & PhD student, PostDoc/Prof/\ldots & thesis writing \\
  &&& / teaching \\
  &&& / name some CP work \ldots \\ 
  \textcolor{red}{A more involved task}      &    &    &  \\ 
  \bottomrule
  
  \multicolumn{4}{l}{\textbf{Describe a second milestone}} \\
  \midrule
  First task \ldots      &          &  &  \\ 
  \bottomrule
\end{tabular}
%}
\end{table}




\section{Object selection}
The supporting notes should now include the following standardized tables of properties: each analysis should simply fill them
in by writing / replacing the value with the appropriate number or by choosing the appropriate option.
The idea of these tables is to harmonize some sections of the supporting notes as to make review and analysis comparisons simpler.

If you use non-standard selections which do not fit in these tables, this should of course be noted and discussed in more detail in the text.
 
\include{object_selection/egamma_selection}
\include{object_selection/muon_selection}
\include{object_selection/tau_selection}
\subsection{Small-R jet selection}

\emph{Note: these tables still have to be adapted to normal ATLAS conventions.}

\begin{table}[ht]
\resizebox{\textwidth}{!}{
\begin{tabular}{|c|c|}
\hline
%\large
\multicolumn{2}{|c|}{Jet reconstruction parameters} \\
%\normalsize
\hline
Parameter & Value \\ 
\hline
algorithm & anti-k$_{T}$  \\
R-parameter & 0.4 \\
input constituent & EMTopo \\
Analysis Release Number & 21.2.10 \\
%Calibration tag & JetCalibTools-00-04-76 \\
CalibArea tag & 00-04-81 \\
Calibration configuration & JES\_data2017\_2016\_2015\_Recommendation\_Feb2018\_rel21.config \\
Calibration sequence (Data) & JetArea\_Residual\_EtaJES\_GSC\_Insitu \\
Calibration sequence (MC) & JetArea\_Residual\_EtaJES\_GSC \\
%Calibration sequence (AFII) & JetArea\_Residual\_EtaJES\_GSC \\
\hline
%\large
\multicolumn{2}{|c|}{Selection requirements} \\
%\normalsize
\hline
Observable & Requirement \\
\hline
Jet cleaning & LooseBad \\
BatMan cleaning & No \\
pT  & $>$XX GeV \\
\textbar$\eta$\textbar & $<$X \\
JVT & (update if needed) $>$0.59 for $p_{T}<$60 GeV , \textbar$\eta$\textbar$<$2.4\\
\hline
\end{tabular}}
\end{table}


\clearpage
\subsection{Large-R jet selection}

\begin{table}[ht]
\resizebox{\textwidth}{!}{
\begin{tabular}{|c|c|}
\hline
%\large
\multicolumn{2}{|c|}{Jet reconstruction parameters} \\
\hline
%\normalsize
Parameter & Value \\ 
\hline
algorithm & anti-k$_{T}$  \\
R-parameter & 1.0 \\
input constituent & LCTopo \\
grooming algorithm & Trimming \\ 
$f_{cut}$ & 0.05 \\
$R_{trim}$ & 0.2 \\
Analysis Release Number & 21.2.10 \\
%Calibration tag & JetCalibTools-00-04-76 \\
CalibArea tag & 00-04-81 \\
Calibration configuration & JES\_MC16recommendation\_FatJet\_JMS\_comb\_19Jan2018.config \\
Calibration sequence (Data) & EtaJES\_JMS\_Insitu \\
Calibration sequence (MC) & EtaJES\_JMS \\
\hline
%\large
\multicolumn{2}{|c|}{Selection requirements} \\
%\normalsize
\hline
Observable & Requirement \\
\hline
pT  & $>$X GeV \\
\textbar$\eta$\textbar & $<$X \\
mass & $>$ X GeV \\
\hline
\multicolumn{2}{|c|}{Boosted Object Tagger} \\\hline
Object  & Working point \\\hline
$W$ / $Z$ / Top & 50\% / 80\% \\
$X\rightarrow bb$ & single/double b-tag with/without loose/tight mass \\\hline

\end{tabular}}
\end{table}

\subsection{MET selection}


\begin{table}[ht]
\begin{center}
\begin{tabular}{|c|c|}
\hline
%\large
\multicolumn{2}{|c|}{MET reconstruction parameters} \\
\hline
%\normalsize
Parameter & Value \\ 
\hline
Algorithm & Calo-based \\
Soft term & Track-based (TST) \\ 
MET operating point & Tight \\
Analysis release & 21.2.16 \\
Calibration tag & METUtilities-00-02-46 \\
\hline
%\large
\multicolumn{2}{|c|}{Selection requirements} \\
%\normalsize
\hline
Observable & Requirement \\
\hline
$E_{T}^{miss}$  & $>$X GeV \\
$\frac{\sum{E_{T}}}{E_{T}^{miss}}$  & $<$X \\
Object-based $E_{T}^{miss}$ significance & $>$ X \\

\hline
\end{tabular}
\end{center}
\end{table}



\subsection{Jet flavour tagging selection}


\begin{table}[ht]
\begin{center}
\resizebox{\textwidth}{!}{
\begin{tabular}{|l|c|}\hline
  \multicolumn{2}{|c|}{b-tagging selection} \\\hline

				& EM Topo Jets / Track jets / VR jets \\\hline\hline
Jet collection		& AntiKt4EMTopo / AntiKt2PV0 / AntiKtVR30Rmax4Rmin02 \\\hline
Jet selection 		& $\pT >$ X GeV    \\				            	& $|\eta| <$ Y \\				
                 & $JVT$ cut if applicable \\
                 \hline
Algorithm 		& MV2c10 / MV2c10mu / MV2c10rnn / DL1 / DL1mu /DL1rnn 	\\\hline
Operating point		& Hybrid /  Fixed \\
                     & Eff = 60 / 70 / 77 / 85 
   \\\hline                  
CDI                & 2017-21-13TeV-MC16-CDI-2017-12-22\_v1 \\\hline
\end{tabular}}
\end{center}
\label{tab:bstar}
\end{table}%


\subsection{Tracks selection}

If you use tracks as particular objects on which you cut in your analysis.

\begin{table}[h!]
\centering
\resizebox{\textwidth}{!}{
\begin{tabular}{|l|c|}
\hline
  \multicolumn{2}{|c|}{TrackParticle object selection} \\\hline

Tracking Algorithm								    & Primary/Large Radius Tracking/Custom \\
Track Quality Selection (official)      & Loose/Tight                          \\
Additional Selections                    &                                      \\
\ $|\eta|$                                          & $<$X                           		 \\
\ $p_{T}$                                        & $>$X~GeV                           \\
Track-Vertex-Association Criteria        & Loose/Tight                          \\
Track-to-Jet Association Method         & Ghost Matched/dR                     \\       \hline  
\end{tabular}}
\end{table}

\subsection{Overlap Removal}

\emph{Note: this table still have to be adapted to normal ATLAS conventions.}

The reconstruction of the same energy deposits as multiple objects is resolved using the standard overlap removal tools, AssociationUtils, documented \href{https://gitlab.cern.ch/atlas/athena/blob/21.2/PhysicsAnalysis/AnalysisCommon/AssociationUtils/README.rst}{here}

The (Standard/Heavy-flavor/Boosted/Boosted+Heavy-flavor/Lapton-favored) working point is used corresponding to:

\begin{table}[ht]
\begin{center}
\small
\resizebox{\textwidth}{!}{
\begin{tabular}{|c|c|c|}
\hline
 Reject & Against & Criteria \\\hline
 electron & electron & shared track, $p_{T,1} < p_{T,2}$ \\
 tau      & electron & $\Delta R <$ 0.2 \\
 tau      & muon     & $\Delta R <$ 0.2 \\
 muon     & electron & is calo-muon and shared ID track \\
 electron & muon     & shared ID track \\
 photon   & electron & $\Delta R <$ 0.4 \\
 photon   & muon     & $\Delta R <$ 0.4 \\
 jet      & electron & [$\Delta R <$ 0.2/Not a bjet and $\Delta R <$ 0.2] \\
 electron & jet      & [$\Delta R <$ 0.4/$\Delta R <$ min(0.4, 0.04 + 10GeV/ElePt)/None] \\
 jet      & muon     & [NumTrack $<$ 3 and (ghost-associated or $\Delta R <$ 0.2) \\
  && / Not a bjet and NumTrack $<$ 3 and (ghost-associated or $\Delta R <$ 0.2)] \\
 muon     & jet      & [$\Delta R <$ 0.4/$\Delta R <$ min(0.4, 0.04 + 10GeV/MuPt)/None] \\
 jet      & tau      & $\Delta R <$ 0.2 \\
 photon   & jet      & $\Delta R <$ 0.4 \\
 fat-jet  & electron & $\Delta R <$ 1.0 \\
 jet      & fat-jet  & $\Delta R <$ 1.0 \\
\hline
\end{tabular}}
\end{center}
\end{table}

$\Delta R$ is calculated using rapidity by default.





\section{Event selection}
The following items should also be filled in for the event selection.

\include{event_selection/event_cleaning}

\end{document}
